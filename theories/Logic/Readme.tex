\PassOptionsToPackage{dvipsnames}{xcolor}
\PassOptionsToPackage{top=2cm,bottom=2cm}{geometry}
\documentclass[10pt]{acmart}
\usepackage{proof}

\title{Ctrees,sets of ctrees and CTL}

\begin{document}

%%% CTREE
\section*{Ctrees}

We define ctrees and labels the usual way
\begin{align*}
  \texttt{ctree\ t, u} := &\ \texttt{Vis e k} \\
           &\ \texttt{Br (b:bool) c k} \\
           &\ \texttt{Ret x} \\
  \texttt{label \ l} := &\ \texttt{obs e x} \\
           &\ \texttt{tau} \\
           &\ \texttt{val x}
\end{align*}

And alias $\texttt{stuck} := \texttt{Br false}\ c_0\ (\texttt{match \_ with end})$.\\

Then we define the transition relation \texttt{trans: ctree -> label -> ctree} between ctrees.\\

\begin{minipage}{0.4\textwidth}
\infer{
  \texttt{trans}\ (\texttt{Vis e k})\ (\texttt{obs e x})\ \texttt{(k x)}
}{}
\vspace*{1cm}
\infer{
  \texttt{trans}\ (\texttt{Ret x})\ (\texttt{val x})\ \texttt{stuck}
}{}
\end{minipage}
\hspace{2cm}
\begin{minipage}{0.4\textwidth}

\infer{
  \texttt{trans}\ (\texttt{Br true c k})\ (\texttt{tau})\ \texttt{(k x)}
}{}
\vspace*{1cm}
\infer{
  \texttt{trans}\ (\texttt{Br false c k})\ l\ (\texttt{t})
}{
  \exists x,\ \texttt{trans}\ \texttt{(k x)}\ l\ \texttt{t}
}

\end{minipage}
\vspace{1cm}

\section*{Kripke semantics}
A Kripke model $K$ is a 5-tuple $(W,\ I,\ \to,\ AP,\ L)$.
\begin{enumerate}
\item $W$ is a set of worlds.
\item $I$ is the initial world $I \in W$.
\item $\mapsto$ is the accessibility relation $\to \subseteq W \times W$.
\item $AP$ is the set of atomic predicates
\item $L: W \to \mathcal{P}(AP)$ is a state labeling function.
\end{enumerate}

\section*{Ctrees to Kripke model}
We only consider \texttt{ctree E C void} that never return.
We pick $(S: Type)$ and initial value $(s: S)$. Pick a total function $h:\ \forall (X: Type),\ E X \to S \to S$.
Now construct a Kripke model $K$ from a ctree $t$ 
\begin{enumerate}
\item $W : (ctree * S)$.
\item $I := (t, s)$.
\item $\mapsto$ is inductively defined below.
\item $AP : (ctree * S)$.
\item $L := \lambda x. \{ x \} $.
\end{enumerate}
\vspace{1cm}
\begin{minipage}{0.3\textwidth}
\infer{
  (\texttt{Br true c k},\ s)\ \mapsto \ (k\ x, s)
}{
  \texttt{trans}\ \texttt{(Br true c k)}\ \texttt{tau}\ \texttt{(k x)}
}
\end{minipage}
\hspace{1cm}
\begin{minipage}{0.3\textwidth}
\infer{
  (\texttt{Vis e k},\ s)\ \mapsto \ (k\ x, h\ e\ s)
}{
  \texttt{trans}\ \texttt{(Vis e k)}\ \texttt{(obs e x)}\ \texttt{(k x)}
}
\end{minipage}

%%%% CTL
\newpage
\section*{Ctl}
We define the syntax of CTL formulas the following way.
$$ P\ :\ S \to Prop $$
\begin{align*}
  Ctl\ p, q\ := &\ P \\
  &\ p \& q \\
  &\ p \| q \\
  &\ p \Rightarrow q \\
  &\ \texttt{AX}\ p \\
  &\ \texttt{EX}\ p \\
  &\ \texttt{AG}\ p \\
  &\ \texttt{EG}\ p \\
  &\ \texttt{AF}\ p \\
  &\ \texttt{EF}\ p \\
\end{align*}

Then we define a denotation of Ctl formulas to a set of $(ctree, S)$ elements, or $ctree \to S \to Prop$. Here is the
denotation function $\llbracket - \rrbracket: Ctl \to ctree \to S \to Prop $

\begin{align*}
  \llbracket P \rrbracket := &\ \lambda\ t\ s.\ P\ s \\
  \llbracket p \& q \rrbracket := &\ \lambda\ t\ s.\ \llbracket p \rrbracket\  t\ s \land \llbracket q \rrbracket t\ s \\
  \llbracket p \| q \rrbracket := &\ \lambda\ t\ s.\ \llbracket p \rrbracket\ t\ s \lor \llbracket q \rrbracket t\ s \\
  \llbracket p \Rightarrow q \rrbracket := &\ \lambda\ t\ s.\ \llbracket p \rrbracket t\ s \to \llbracket q \rrbracket t\ s \\
  \llbracket AX\ p \rrbracket := &\ \lambda\ t\ s.\ \forall\ s'\ t',\ (t,s) \mapsto (t',s') \to \llbracket p \rrbracket t'\ s' \\
  \llbracket EX\ p \rrbracket := &\ \lambda\ t\ s.\ \exists\ s'\ t',\ (t,s) \mapsto (t',s') \land \llbracket p \rrbracket t'\ s' \\
  \llbracket AF\ p \rrbracket := &\ \lambda\ t\ s.\mu X. \ \llbracket p \rrbracket\ t\ s \lor
            (\forall\ s'\ t',\ (t,s) \mapsto (t',s') \ \to \llbracket X\ p \rrbracket t'\ s') \\
  \llbracket EF\ p \rrbracket := &\ \lambda\ t\ s.\mu X. \ \llbracket p \rrbracket\ t\ s \lor
            (\exists\ s'\ t',\ (t,s) \mapsto (t',s') \ \land \llbracket X\ p \rrbracket t'\ s') \\
  \llbracket AG\ p \rrbracket := &\ \lambda\ t\ s.\nu X. \ \llbracket p \rrbracket\ t\ s \land
            (\forall\ s'\ t',\ (t,s) \mapsto (t',s') \ \to \llbracket X\ p \rrbracket t'\ s') \\
  \llbracket EG\ p \rrbracket := &\ \lambda\ t\ s.\nu X. \ \llbracket p \rrbracket\ t\ s \land
            (\exists\ s'\ t',\ (t,s) \mapsto (t',s') \ \land \llbracket X\ p \rrbracket\ t'\ s') \\
\end{align*}
Note $AG, EG$ are defined as greatest fixpoints, $AF, EF$ as least fixpoints.

\section*{Satisfiability}
Given a type $S$, ctree $t$, a formula $p$ an initial state $(s: S)$ we can now define the satisfiability relation.

\subsection*{Via Ctl denotation}
$$t,\ s\ \vDash_D p\ :=\ \llbracket p \rrbracket\ t\ s$$

\end{document}
